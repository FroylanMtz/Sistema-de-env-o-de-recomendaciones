%Presentación del problema
La diabetes es actualmente una enfermedad que afecta globalmente a las personas. Especialmente en México la cultura sobre el tratamiento de esta enfermedad es un aspecto al que las personas no le dan la importancia necesaria, hasta el punto de no visitar al médico frecuentemente y perder el seguimiento de esta enfermedad, además lo agudiza que no se realiza ejercicio seguido o no se lleva una alimentación saludable.
\vspace{0.50cm}

%La idea y literatura van juntitas en este parrafito :DDDDD
Actualmente la academia Mexicana está realizando investigación sobre como la tecnología reciente puede ayudar en la educación de las personas para que tomen cartas sobre este asunto, sin embargo, todas estas investigaciones solo se quedan en artículos y muy pocas de ellas se logran concretar en algo real que ayude a solucionar el problema. Lo que se propone en este articulo es utilizar esta información adicionada con reciente tecnología en desarrollo para mantener a las personas informadas sobre la enfermedad, mantenerlos al día sobre que es lo que se recomienda por parte de sector salud y dar algunas recomendaciones para evitar contraer este padecimiento recomendando buena alimentación o actividad física. Se pretende diseñar una aplicación Web que permite enviar alertas y recibir respuestas de los usuarios a través de la aplicación de mensajería instantánea Telegram. A través de esta comunicación bidireccional el sistema pude enviar recomendaciones personalizadas a las personas así como darles seguimiento a su respectivo problema.
\vspace{0.50cm}

%Herramientas a utilizar
Para ello se pretende hacer uso de varias herramientas o API's que ayuden a la comunicación desde un servidor a las aplicaciones de los usuarios ya sea en dispositivos móviles o computadoras personales. Para la parte de la comunicación del envío de mensajes se pretende usar las bibliotecas Telegraf y Express que ayudan a crear el vinculo entre la aplicación y el servidor donde residirá el agente inteligente, para la parte de procesamiento de lenguaje natural se utilizara la biblioteca de Facebook wit.ai. Para que el agente pueda categorizar a los usuarios y realizar envío de alertas personalizadas dependiendo de su estado se hará uso de algormitos de aprendizaje por refuerzo con la ayuda de un modelo realizado con la biblioteca de  TensorFlow.
\vspace{0.50cm}


%Posibles resultado
El resultado que se pretende lograr es la creación de un chatbot inteligente que pueda responder automáticamente a los usuarios enviando notificaciones a ciertas horas del día sobre recomendaciones alimenticias o sobre la activación física. Previamente recopilando información del usuario para personalizar dichas recomendaciones en base a su estado de salud o sus padecimiento.
\vspace{0.50cm}
