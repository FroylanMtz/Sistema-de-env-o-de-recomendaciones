\section{Introducción}

%identificacion de ideas
%incluir subsecciones, e incluir algunas ideas de las subsecciones

%*Se va a trabajar en la introduccion*

%*Es importante tener indice*

%1 Introduccion: Introduccion al campo, importancia para las personas, que aplicaciones puede tener para los individuos, empresas etc, es donde se justifica el area de estudio y porque atenderla.

La diabetes es una enfermedad a nivel mundial que afecta a todo tipo de poblacion, con el crecimiento de la industria esto se ha agrabado debido a que se consumen cada vez mas productos ultraprocesados que fomentan la aparicion de esta enfermedad.

\vspace{0.50cm}

\subsection{Definicion del problema}
%Definicion del problema: Elegir solo un problema, y dejar subproblemas de manera secundaria, describir porque es importante el problema, tambien se puede hacer una descripcion de lo que se ha hecho con el problema,y cuales son los faltantes, cuales son los aspectos que faltan cubrir con ese problema.


La diabetes es una enfermedad metabólica que se caracteriza por niveles de azúcar en sangre más elevados de lo normal. Esto se produce por un fallo en la producción o acción de la insulina. De no controlarse adecuadamente, a largo plazo, puede provocar alteraciones en riñones, corazón u ojos.

\vspace{0.50cm}

Existen dos tipos principales de diabetes:
\vspace{0.50cm}

La tipo 1, que es una de las enfermedades crónicas infantiles y se produce por un factor genético, es decir, un familiar tiene la enfermedad y se hereda o por autoinmunidad. En ella el páncreas no fabrica la insulina suficiente.
\vspace{0.50cm}

La tipo 2, más frecuente en las personas mayores. En este caso la capacidad de producir insulina no desaparece pero el cuerpo presenta resistencia a esta hormona. También puede ser hereditaria, aunque la mayoría de las personas la sufren por el estilo de vida que llevan: alimentación poco saludable, personas con exceso de peso o estilo de vida sedentario, por ejemplo.
\vspace{0.50cm}

En esta última causa, el papel de la prevención es fundamental. Por eso es importante tener controlado nuestro peso, mantenerse activo con un ejercicio regular de al menos 30 minutos diarios, cuidar nuestra alimentación y comer de una forma sana, descansar y dejar que el cuerpo se recupere durmiendo bien.
\vspace{0.50cm}

La alta produccion de comida procesada en el mundo es un punto que empeora la situacion de la enfermedad, debido a que las personas hoy en dia consumen alimentos con altos contenidos caloricos y procesados.
\vspace{0.50cm}

Ademas de ser un problema para la sociedad civil, la diabetes es un problema que afecta a la economia de los paises y en las cifras recientes se tiene que la diabetes cuesta al gobierno un total de X millones de pesos, todo por que la gente no sabe cuidarse correctamente.
\vspace{0.50cm}

Ademas de esto se tienen estadisticas que muestran la baja educacion nutricional por parte de los padres a sus hijos lo que cause la mala alimentacion, poco ejercicio, y sedentarismo. Hoy en dia un niño promedio Mexicano prefiere un videojuego antes que un balon de futbol soccer.
\vspace{0.50cm}

Tambien se tiene por otro lado que cada dia mas Mexicanos estan obteniendo una smarthphone con el que peuden comunicarse instantaneamente con otras personas gracias a aplicaciones moviles de mensajeria instantanea y la expacion de informacion rapida como las redes sociales.
\vspace{0.50cm}




\subsection{Objetivos generales y especificos} 
%No es necesario que haya objetivos especificos, si se requiere objetivo general. Se pueden pensar como metas que se tienen que alcanzar par lograr cumplir con el objetivo general, estas metas pueden estar en secuencia, o se peude pensar como objetivos generales que no necesariamente se tienen que cumplir en orden, consebirlos como una secuencia de objetivos.

\textbf{Objetivo General}

Lo que se pretende en el presente proyecto es utilizar la tecnologia actual para desarrollar una herramienta la cual permita enviar recomendaciones a su dispositivo movil para la concientizacion de la diabetes en personas adultas que padecen alguno de los problemas que ayudan a generar la diabetes.
\vspace{0.50cm}

Esta herramienta automatizará el envio de mensajes a los usuarios, pero ademas de ello personalizara éstos al pedirle informacion inicial al usuario para "categorizarlo en un estado", es decir, no todos los usuarios estaran en la misma situacion de salud y por ende cada uno recibira un mensaje personalizado de lo que debe ralizar para mejorar alimentacion y prvenir la diabetes. Para poder realizar esta personalizacion de los mensajes se prentende usar el principio de los sitemas de recomendacion, es decir un modelo de Aprendizaje por Refuerzo que usa los procesos de decisiones de Markov (MDP) para realizar un inteligente seleccion de los mensajes que hay que enviar al usuario.
\vspace{0.50cm}


\vspace{0.50cm}


\textbf{Objetivos Especificos}

Algunos objetivos que se pretenden alcanzar para juntos lograr el principal son los siguientes:

\begin{itemize}
    \item Creacion de un servidor que contendra toda la funcionalidad de la herramienta, para que esté activo en internet
	\item Realizacion de una herramienta de comunicación a traves de respuestas http con la aplicacion Telegram.
	\item Envio de respuestas y peticiones en formato JSON
	\item Respuesta a los usuarios a traves de la aplicacion de Telegram, ya sea en PC o en dispositivos moviles
	\item Procesamiento del lenguaje natural para descifrar la peticion de lo que el usuario quiere saber
	\item Creacion de un modelo inteligente ML para el sistema para seleccionar la mejor recomendacion para el usuario
\end{itemize}

\subsection{Alcances y Limitaciones}
%Esto es la definicion hasta donde se va a llegar y hasta donde no, que se va ahcer que no se va ahacer, que se va a considerar, que no se va a considerar. Esto sirve para definir el alcance del proyecto. Todos los detalles del alcance van en esta parte. se encuadra el alcance. Se especifican las evaluaciones.

El alcance del proyecto es hasta un cierto punto muy ambicioso, pero lamentablemente no se puede realizar, estamos perdidos no sabes como integrar el modelo con el resto de los modulos y no poseemos los suficientes datos de entrenamiento para el modulo del procesamiento del lenguaje natural, oh dios todo es un fiasco, pero saldremso avantes en la incesable lucha de poder tener un promedio que nso ayude a salir de esto, el dia de mañana lo define el tiempo y el destino.

It's really hard for me to admit it, but i'm afraid of people. It's because i've had very bad experiences with them when i was a kid. It feels like no matter how hard i'm trying, i'll always be a loser. I'm scared when i have to speak. People thinks i have no emotions, but i have really strong emotions. I just don't want to show them, because others will just use me or laugh to me... It feels like i lost myself and i can't be normal, sometimes i don't even feel like a human. I just want to end this. I don't wanna live like this anymore. I hope someone is out there in world to make me feel better again. It's the only hope that keeps me alive.

\subsection{Contenido del documento}
%es la ultima subseccion de la seccion de introduccion. aqui se describe lo que se va a escribir en el resto del documento. Descripcion del marco teorico. Estado del arte.

El contenido no es mas que una simple vaga idea de lo que en realidad se hara, lo digo porque no tenemos la sensibilidad de entender un proyecto ni aunque nos golpee la cabeza, todo esto se hara de la forma incorrecta y la forma que nadie lo realiza, un total asco para la academia y una aberracion que solo nos dara mediocridad con cada palabra que seguimos escribiendo.