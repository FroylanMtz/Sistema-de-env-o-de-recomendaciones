\section{Conclusiones}

El creciente desarrollo de la computacion ha tenido gran auge debido al poder computacional con el que hoy en dia cuentan nuestros dispositivos. Pero aun asi es corto en comparacion con los problemas aun que nos queda por resolver, problemas que en ocasiones no tienen una solucion en un espacio temporal adecuado.

Afortunadamente los programadores y cientificos de datos estan ideando nuevos metodos para hacerles frente a estos problemas, o para encontrar una solucion B que se acerque al optimo requerido, algunas de estas tecnicas son las de aprendizaje automatico, en las cuales se programan a las computadoras para que estas puedan aprender en base a su experiencia emulando los procesos en los seres vivos.

Como se vio en el presente documento se aplico una solucion relativamente sencilla a un problema en un entorno controlado gracias a la biblioteca gym que pertenece a la organizacion de OpenAI, en el cual se aplicaron algoritmos de aprendizaje por refuerzo y mas puntualmente el algoritmo de Q-Learning. Ademas de esto se aplicaron metodos complejos como lo son las redes neuronales para optimizar la solucion y poder resolverlo en un espacio temporal y espacial mucho mas corto y ademas escalable si el problema crece comparados con los metodos tradicionales de Q-Learning.

Definitivament el aprednizaje automatico y la inteligencia artificial junto con metodos complejos hacen la diferencia ante distintos problemas computacionales a los cuales hoy en dia muchos programadores y cientificos se enfrentan.